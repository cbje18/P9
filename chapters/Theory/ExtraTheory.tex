Let us give a brief overview of the basic set-up, when we are dealing with financial markets and derivatives pricing using the no-arbitrage principle in a continuous time setting. Firstly, we start off with a definition of a financial market.
\begin{defn}[\textit{Financial Market}]
    For a fixed $T>0$, a finite-horizon financial market is a pair
    \begin{equation}
        \mathfrak{M}=\left\{\left(\Omega,\mathcal{F},(\mathcal{F}_{t})_{0\leq t\leq T},\hspace{2 pt}\mathbb{P}\right), P=\left(A_{t},S_{t}^{(1)},\dots,S_{t}^{(d)}\right)_{0\leq t\leq T}\right\},
    \end{equation}
    where $(\Omega,\mathcal{F},(\mathcal{F}_{t})_{0\leq t\leq T},\hspace{2 pt}\mathbb{P})$ is a filtered probability space satisfying the usual assumptions of right-continuity and completeness, and $P$ is a collection of $d+1\in\N$ continuous semimartingales.
\end{defn}
Note that a filtered probability space is just probability space equipped with a filtration, i.e. a family $\left(\mathcal{F}_{t}\right)_{t\in I}$ of sub-$\sigma$-algebras of $\mathcal{F}$ satisfying $\mathcal{F}_{s}\subset \mathcal{F}_{t}\subset\mathcal{F}$ for all $s,t\in I$ with $s\leq t$. The assumption of right-continuity is that $\mathcal{F}_{t}=\mathcal{F}_{t^{+}}$ for all $t\in [0,\infty)$, where
\begin{equation}
    \mathcal{F}_{t^{+}}\coloneqq\bigcap_{s>t}\mathcal{F}_{s}.
\end{equation}
A filtered $(\Omega,\mathcal{F},(\mathcal{F}_{t})_{0\leq t\leq T},\hspace{2 pt}\mathbb{P})$ probability space is said to be complete, if $(\Omega,\mathcal{F},\mathbb{P})$ is a complete measure space, and $\mathcal{F}_{0}$ contains all $\mathbb{P}$-zero sets. 

A continuous semimartingale is a process $(X_{t})_{t\geq 0}$ defined on a filtered probability space, which admits the decomposition
\begin{equation}\label{decomp}
    X_{t}=X_{0}+M_{t}+A_{t},\quad t\geq 0,
\end{equation}
where $M\in \mathcal{M}_{c,loc}$ and $A\in \mathcal{A}_{loc}$ with $M_{0}=A_{0}=0$. Such a decomposition is unique up to indistinguishability. $M$ is a local martingale with continuous paths, and $A$ is an adapted process with locally bounded variation.

The process $A$ (not the same $A$ as in \eqref{decomp}) is our numéraire, which measures the value of money through time. It satisfies for all $t\in [0,T]$ that
\begin{equation}
    \mathbb{P}(A_{t}>0)=1.
\end{equation}
\begin{defn}[\textit{Portfolio, Wealth Process}]
    A portfolio (strategy) is any adapted process
    \begin{equation}
        \theta=(\phi_{t},\pi_{t}^{(1)},\dots,\pi_{t}^{(d)})_{0\leq t\leq T}.
    \end{equation}
    The wealth process associated to the portfolio $\theta$ with initial investment $V_{0}\in \mathcal{F}_{0}$ is
    \begin{equation}
        V_{t}^{\theta}=\phi_{t}A_{t}+\sum_{j=1}^{d}\pi_{t}^{(j)}S_{t}^{(j)}, V_{0}^{\theta}=V_{0}, \quad t\in [0,T].
    \end{equation}
\end{defn}
Next, we introduce the class of admissible strategies:
\begin{defn}[\textit{Admissible Strategy}]
    A strategy $\theta=(\phi_{t},\pi_{t}^{(1)},\dots,\pi_{t}^{(d)})_{0\leq t\leq T}$ is said to be admissible if
    \begin{enumerate}
        \item It is self-financed
        \begin{equation}
            dV_{t}^{\theta}=\phi_{t}dA_{t}+\sum_{j=1}^{d}\pi_{t}^{(j)}dS_{t}^{(j)},
        \end{equation}
        i.e. no exogenous capital is injected into the portfolio.
        \item It has a finite credit line, i.e. there exists a non-random constant $C>0$ such that almost surely
        \begin{equation}
            V_{t}^{\theta}>-C, \quad t\in [0,T].
        \end{equation}
    \end{enumerate}
\end{defn}
Now, we are ready to define the very important concept of an arbitrage.
\begin{defn}[\textit{Arbitrage}]
    An admissible strategy $\theta$ is said to be an arbitrage if
    \begin{enumerate}
        \item It has zero initial capital
        \begin{equation}
            V_{0}^{\theta}=\phi_{0}A_{0}+\sum_{j=1}^{d}\theta_{0}^{(j)}S_{0}^{(j)}=0.
        \end{equation}
        \item There is a non-random point in time $t_{0}\in (0,T]$ in which we are out of debts almost surely
        \begin{equation}
            V_{t_{0}}^{\theta}=\phi_{t_{0}}A_{t_{0}}+\sum_{j=1}^{d}\theta_{t_{0}}^{(j)}S_{t_{0}}^{(j)}\geq 0.
        \end{equation}
        \item We have a chance to make a profit at that moment
        \begin{equation}
            \mathbb{P}\left(V_{t_{0}}^{\theta}>0\right)>0.
        \end{equation}
    \end{enumerate}
\end{defn}
In a perfect market there should be no arbitrage opportunities, since this indicates a risk-less chance of profit. Therefore, we want to price derivatives using the no-arbitrage principle, such that every investor, who wants to make a profit, also has to take on some risk. 

The discounted price process relative to the numéraire $A$ is defined by
\begin{equation}
    \Tilde{P}_{t}\coloneqq\frac{P_t}{A_t}, \quad j=0,1,\dots,d+1.
\end{equation}
\begin{defn}[\textit{Martingale Measure}]
    A probability measure $\mathbb{Q}$ on $(\Omega,\mathcal{F}_{T})$ is said to be a (local) martingale measure, if the discounted price process $(\Tilde{P}_{t})_{0\leq t\leq T}$ is a $\mathbb{Q}$ (local) martingale.
\end{defn}
We are now ready to present the First Fundamental Theorem of Asset Pricing.
\begin{thm}
The market $\mathfrak{M}$ is arbitrage free, if there exists a local martingale measure $\mathbb{Q}$, which is equivalent to $\mathbb{P}$.
\end{thm}
Note that this is not an "if and only if"-statement. Hence, we only have to prove that if there exists a local martingale measure $\mathbb{Q}$ equivalent to $\mathbb{P}$, then the market $\mathfrak{M}$ is arbitrage free.
\begin{proof} Let $\theta=(\phi_{t},\pi_{t}^{(1)},\dots,\pi_{t}^{(d)})_{0\leq t\leq T}$ be an admissible strategy such that $V_{0}^{\theta}=0$. Assume that $\mathbb{Q}$ is a martingale measure equivalent to $\mathbb{P}$. Then the discounted wealth process
\begin{equation}
    \Tilde{V}_{t}^{\theta}=\int_{0}^{t}\pi_{s}d\Tilde{S}_{s}, \quad 0\leq t\leq T,
\end{equation}
is local martingale with respect to $\mathbb{Q}$. Since $\theta$ is an admissible strategy, there is a non-random constant $C>0$ such that $V^{\theta}+C$ is a non-negative local martingale. We then use the fact that any continuous non-negative local martingale is a supermartingale. Consequently, we have
\begin{equation}\label{supermart}
    \mathbb{E}_{\mathbb{Q}}(\Tilde{V}_{t}^{\theta})\leq \mathbb{E}_{\mathbb{Q}}(\Tilde{V}_{0}^{\theta})=0, \quad 0\leq t\leq T.
\end{equation}
Let $t_{0}\in [0,T]$ such that $\mathbb{P}(\Tilde{V}_{t_0}^{\theta}\geq 0)=1$. Since $\mathbb{P}$ and $\mathbb{Q}$ are equivalent measures, we have $\mathbb{Q}(\Tilde{V}_{t_0}^{\theta}\geq 0)=1$, whence $\mathbb{E}_{\mathbb{Q}}(\Tilde{V}_{t_{0}}^{\theta})\geq 0$. It now follows from \eqref{supermart} that $\Tilde{V}_{t_{0}}^{\theta}=0$ $\mathbb{P}$ and $\mathbb{Q}$-a.s. Hence, we have shown that an admissible strategy with zero initial capital with which we are out of debts almost surely, will have no chance of making a profit, i.e. the market does not admit an arbitrage.
\end{proof}
We can now define the concept of a European financial derivative. For a financial market $\mathfrak{M}$, we will say that a random variable $\xi$ is a European financial derivative with date of maturity $T>0$, if it is $\mathcal{F}_{T}$-measurable and integrable. Furthermore, $\xi$ is called simple if it only depends on $P_T$, i.e. $\xi=\Phi(P_T)$ for some measurable function $\Phi:\R^{d+1}\to \R$. The function $\Phi$ is called the pay-off function of $\xi$.

We can now price derivatives using the First Fundamental Theorem of Asset Pricing.
\begin{cor}
Consider the market
\begin{equation}
    \mathfrak{M}=\left\{\left(\Omega,\mathcal{F},(\mathcal{F}_{t})_{0\leq t\leq T},\hspace{2 pt}\mathbb{P}\right), P=\left(A_{t},S_{t}^{(1)},\dots,S_{t}^{(d)}\right)_{0\leq t\leq T}\right\}.
\end{equation}
Let $\xi$ be a European financial derivative. Then the extended market
\begin{equation}
    \mathfrak{M}=\left\{\left(\Omega,\mathcal{F},(\mathcal{F}_{t})_{0\leq t\leq T},\hspace{2 pt}\mathbb{P}\right), P=\left(A_{t},S_{t}^{(1)},\dots,S_{t}^{(d)}, \xi_{t}\right)_{0\leq t\leq T}\right\}
\end{equation}
is arbitrage free, if there is a local martingale measure $\mathbb{Q}$ equivalent to $\mathbb{P}$ such that
\begin{equation}
    \xi_{t}=\mathbb{E}_{\mathbb{Q}}\left(\frac{A_t}{A_T}\xi\mid \mathcal{F}_{t}\right),\quad 0\leq t\leq T.
\end{equation}
\end{cor}
The First Fundamental Theorem of Asset Pricing thus enables us to price a European derivative. However, it doesn't necessarily ensure that this price is unique. This leads to the Second Fundamental Theorem of Asset Pricing. 
\begin{defn}[\textit{Attainable Derivative, Complete Market}]
Let $\xi$ be an $\mathcal{F}_{T}$-measurable random variable. We say $\xi$ is attainable if there exists $x\in\R$ and an admissible strategy $\theta = (\phi_{t},\pi_{t}^{(1)},\dots, \pi_{t}^{(d)})_{0\leq t\leq T}$ such that the wealth process $(V_{t}^{\theta})_{0\leq t\leq T}$ is bounded and
\begin{equation}
    \frac{\xi}{A_{T}}=\Tilde{V}_{T}^{\theta}=x+\sum_{j=1}^{d}\int_{0}^{T}\pi_{s}^{(j)}d\Tilde{S}_{s}^{(j)}.
\end{equation}
We say a financial market $\mathfrak{M}$ is complete if every $\mathcal{F}_T$-measurable bounded random variable is attained.
\end{defn}
We can now present the Second Fundamental Theorem of Asset Pricing.
\begin{thm}
Suppose there exists a local martingale measure $\mathbb{Q}$ equivalent to $\mathbb{P}$. If the market is complete, then $\mathbb{Q}$ is the unique local martingale measure, which is equivalent to $\mathbb{P}$.
\end{thm}
\begin{proof}
    Let $\mathbb{Q},\mathbb{Q}'$ be two local martingale measures equivalent to $\mathbb{P}$, and let $\xi$ be $\mathcal{F}_T$-measurable and bounded. Put
    \begin{equation}
        \frac{d\mathbb{Q}}{d\mathbb{P}}=Z\in\mathcal{F}_{T},\quad \frac{d\mathbb{Q}'}{d\mathbb{P}}=Z' \in \mathcal{F}_{T}.
    \end{equation}
    Since the market is complete, there exists $\theta = (\phi_{t},\pi_{t}^{(1)},\dots,\pi_{t}^{(d)})_{0\leq t\leq T}$ such that the wealth process $(V_{t}^{\theta})_{0\leq t\leq T}$ is bounded and
    \begin{equation}
        \frac{\xi}{A_{T}}=\Tilde{V}_{T}^{\theta}=x+\sum_{j=1}^{d}\int_{0}^{T}\pi_{s}^{(j)}d\Tilde{S}_{s}^{(j)}.
    \end{equation}
    Furthermore, since $\mathbb{Q}$ and $\mathbb{Q}'$ are local martingale measures, we deduce that the wealth process is a $\mathbb{Q},\mathbb{Q}'$-local martingale. A bounded local martingale is a true martingale, and thus
    \begin{equation}
        \mathbb{E}_{\mathbb{P}}\left(Z\frac{\xi}{A_T}\right)=\mathbb{E}_{\mathbb{Q}}\left(\frac{\xi}{A_T}\right)=x=\mathbb{E}_{\mathbb{Q}'}\left(\frac{\xi}{A_T}\right)=\mathbb{E}_{\mathbb{P}}\left(Z'\frac{\xi}{A_T}\right),
    \end{equation}
    or 
    \begin{equation}
        \mathbb{E}_{\mathbb{P}}\left(\frac{Z-Z'}{A_T}\xi\right)=0,
    \end{equation}
    for every bounded $\mathcal{F}_T$-measurable random variable. Consequently
    \begin{equation}
        \frac{Z-Z'}{A_T}=0, \mathbb{P}-a.s.
    \end{equation}
Since $A$ is numéraire, then necessarily $Z=Z'$ $\mathbb{P}$ almost surely.
\end{proof}