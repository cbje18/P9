\chapter{Preliminary Theory}
In this chapter, we will.....

\section{Arbitrage Theory}

\section{Stochastic Differential Equations}\label{section:SDE}
A general stochastic differential equation (SDE) is a $d$-dimensional system of the form
\begin{equation}\label{integral}
    X_{t}^{(i)}=X_{0}^{(i)}+ \int_{0}^{t}b^{(i)}(s,X_{s})ds+\sum_{j=1}^{m}\int_{0}^{t}\sigma^{(i,j)}(s,X_{s})dB_{s}^{(j)},\quad t\geq 0,
\end{equation}
where $b^{(i)}:[0,\infty)\times \R^{d}\to \R$, $\sigma^{(i,j)}:[0,\infty)\times\R^{d}\to\R$ for $i=1,\dots,d$ and $j=1,\dots,m$ are two families of Borelian functions, and $B$ is a $m$-dimensional Brownian motion. Equation \eqref{integral} is a SDE expressed in its integral form. Equivalently, it could also have been written as
\begin{equation}
    dX_{t}^{(i)}= b^{(i)}(t,X_{t})dt +\sigma^{(i,j)}(t,X_{t})dB_{t}^{(j)}, \quad t\geq 0.
\end{equation}
Recall that a solution of a stochastic differential equation is a triplet of the form \newline$\left((\Omega,\mathcal{F},(\mathcal{F}_{t})_{t\geq 0},\mathbb{P}),B,X\right)$ consisting of a filtered probability space satisfying the usual conditions, a $\R^{m}$-valued Brownian motion $B$ adapted to $(\mathcal{F}_{t})_{t\geq 0}$, and the $(\mathcal{F}_{t})_{t\geq 0}$-adapted $\R^{d}$-valued continuous solution process $X$.

A solution to a SDE, say $\left((\Omega,\mathcal{F},(\mathcal{F}_{t})_{t\geq 0},\mathbb{P}),B,X\right)$, is said to be a strong solution if $X$ is $(\mathcal{F}_{t}^{B})_{t\geq 0}$-adapted, where 
\begin{equation}
    \mathcal{F}_{t}^{B}\coloneqq \sigma\left(\left\{B_{s}^{-1}(A)\mid A\in\mathcal{F}, s\in [0,t] \right\}\right),\quad t\geq 0.
\end{equation}
A solution, which is not strong, is said to be weak. 

Analogous to the theory for ordinary differential equations, we have that if the family of functions $\sigma=(\sigma^{(i,j)})_{1\leq i\leq d, 1\leq j\leq m}$ and $b=(b^{(i)})_{1\leq i\leq d}$ satisfy a Lipschitz-condition in the spatial variable, then existence and uniqueness is guaranteed. More precisely, there is a constant $K>0$ such that
\begin{equation}\label{lipschitz}
    \lVert \sigma(t,x)-\sigma(t,y)\rVert+ \lVert b(t,x)-b(t,y)\rVert \leq K\lVert x-y\rVert
\end{equation}
holds for all $x,y\in\R$ and $t\geq 0$. The existence and uniqueness result is stated in the following.
\begin{thm}\label{existanduniqthm} Let $\sigma=(\sigma^{(i,j)})_{1\leq i\leq d, 1\leq j\leq m}$ and $b=(b^{(i)})_{1\leq i\leq d}$ satisfy the condition \eqref{lipschitz}. Then strong uniqueness holds for the SDE defined by $\sigma$ and $b$. Furthermore, for every probability space $(\Omega,\mathcal{F},(\mathcal{F}_{t})_{t\geq 0},\mathbb{P})$ supporting an $(\mathcal{F}_{t})_{t\geq 0}$-Brownian motion $B$ and every $x\in \R^d$, there exists a unique solution $(B,X^{x})$ such that $X_{0}=x$. Finally, there is a $a>0$ such that
\begin{equation}
    \sup_{t\geq 0}\left\{e^{-at}\sqrt{\mathbb{E}\left[\sup_{0\leq s\leq t}(X_{s}^{x}-X_{t}^{y})^{2}\right]}\right\}\leq C(a,K)\lVert x-y\rVert, \quad \forall x,y\in\R^{d}.
\end{equation}
\end{thm}
The proof of Theorem \ref{existanduniqthm} is omitted.
\subsection{Simulation of SDEs}
In general, finding the solution of a given SDE is a non-trivial task. Consequently, we may turn to numerical simulations of the solution process $X$ of the SDE
\begin{equation}
    X_{t}=x_{0} + \int_{0}^{t}b(s,X_{s})ds + \int_{0}^{t}\sigma(s,X_{s})dB_{s}, \quad t\geq 0,
\end{equation}
where $x_{0}\in \R^{d}$, $B$ is a $m$-dimensional Brownian motion, and the families $b$ and $\sigma$ are as in Section \ref{section:SDE}. Since it is only possible to simulate a finite number of points, we will adopt the notation $X_{t_{0}}^{N},X_{t_{1}}^{N},\dots,X_{t_{N}}^{N}$ for a simulation of $X$ at times $0=t_{0}<t_{1}<\dots <t_{N}=T$ with $T>0$ and $N\in \N$. Suppose, we wish to find an approximation of the solution process $X$ that can be represented as
\begin{equation}\label{simulation}
    X_{0}^{N}=x_{0},\quad X_{t_{k}}^{N}=\Psi_{k,N}(X_{0},\Delta_{1}^{N}B,\dots,\Delta_{k}^{N}B), \quad k=1,\dots,N,
\end{equation}
for some measurable function $\Psi_{k,N}:\R^{d}\times \R^{m\times k}\to \R^{d}$, where 
\begin{equation}
    \Delta_{k}^{N}B\coloneqq B_{k\Delta_{N}}-B_{(k-1)\Delta_{N}}, \quad \Delta_{N}\coloneqq T/N, \quad k=1,\dots, N.
\end{equation}
For $N$ sufficiently large, $X_{t_{k}}^{N}$ will approximate the solution $X_{t_{k}}$ well. Hence, if we have a simulation scheme of the form \eqref{simulation}, then we just need to simulate the path of a Brownian motion at $0=t_{0}<t_{1}<\dots<t_{N}=T$. In order to simulate a Brownian motion, let $(\zeta_{1},\dots,\zeta_{N})$ be a collection of i.i.d. $m$-dimensional random vectors with $\zeta_{k}\sim \mathcal{N}(0,I_{m})$, where $I_m$ is the $m\times m$ identity matrix. If we utilize the independence of the increments of $B$, we get
\begin{equation}
    B_{t_{k}}-B_{t_{k-1}}\overset{d}{=} \sqrt{t_{k}-t_{k-1}}\zeta_{k},\quad k=1,\dots,N.
\end{equation}
Finally, we obtain that
\begin{align}
    (B_{t_{1}},B_{t_{2}},\dots,B_{t_{N}})&= (B_{t_{1}}-B_{t_{0}}, \sum_{k=1}^{2}(B_{t_{k}}-B_{t_{k-1}}),\dots, \sum_{k=1}^{N}(B_{t_{k}}-B_{t_{k-1}}))\\
    &\overset{d}{=}(\sqrt{t_{1}-t_{0}}\zeta_{1},\sum_{k=1}^{2}\sqrt{t_{k}-t_{k-1}}\zeta_{k},\dots,\sum_{k=1}^{N}\sqrt{t_{k}-t_{k-1}}\zeta_{k}).
\end{align}
We summarize in the following algorithm.
\begin{algorithm}[H]
\SetAlgoLined
    \textbf{Input:} $m\in\N, 0=t_{0}<t_{1}<\dots <t_{N}=T$.\newline
    \textbf{Output:} Simulated points $(B_{t_{k}})_{k=0}^{N}$.\\
    $B_{0}\gets 0$; \Comment{Initialization}\\
    \For{$k=1 \textbf{\textrm{ to }} N$}
    {$\zeta_{k} \gets \mathcal{N}(0,I_{m})$;\\
    $B_{t_{k}}\gets B_{t_{k-1}} + \sqrt{t_{k}-t_{k-1}}\zeta_{k}$;}
\caption{Simulation of a Brownian motion.}
\label{alg:BM}
\end{algorithm}
To simulate the solution process $X_{t}$, we will present the Euler-Maruyama scheme. Fix $T>0$ and take $N\in \N$, and define an equidistant partition of $[0,T]$ by $t_{k}=k\Delta_{N}$ for $k=0,1,\dots,N$ with $\Delta_{N}=T/N$. It holds that
\begin{equation}\label{eulerscheme}
    X_{t_{k}}=X_{t_{k-1}} + \int_{t_{k-1}}^{t_{k}}b(s,X_{s})ds + \int_{t_{k-1}}^{t_{k}}\sigma(s,X_{s})dB_{s}, \quad k=1,\dots,N.
\end{equation}
We can make $\Delta_{N}$ as small as we want by choosing $N$ sufficiently large. Consequently, the integrals in \eqref{eulerscheme} can be approximated in the following manner
\begin{equation}
    X_{t_{k}}=X_{t_{k-1}} + b(t_{k-1},X_{t_{k-1}})(t_{k}-t_{k-1}) + \sigma(t_{k-1},X_{t_{k-1}})\Delta_{k}^{N}B + R_{k},
\end{equation}
where
\begin{equation}
    R_{k}\coloneqq \int_{t_{k-1}}^{t_{k}}\left(b(s,X_{s}) - b(t_{k-1},X_{k-1})\right)ds  + \int_{t_{k-1}}^{t_{k}}\left(\sigma(s,X_{s}) - \sigma(t_{k-1},X_{k-1})\right)dB_{s}
\end{equation}
is the residual term for $k=1,\dots, N$. If we neglect this residual term, we obtain the Euler-Maruyama approximation
\begin{equation}\label{eulerrec}
    X_{t_{k}}^{N}=X_{t_{k-1}}^{N} + b(t_{k-1},X_{t_{k-1}}^{N})\Delta_{N} + \sigma(t_{k-1},X_{t_{k-1}}^{N})\Delta_{k}^{N}B,\quad X_{0}^{N}=x_{0}.
\end{equation}
If we iterate the relation \eqref{eulerrec}, we obtain a relation of the form
\begin{equation}
    X_{t_{k}}^{N}=\Psi_{k,N}(X_{0},\Delta_{1}^{N}B,\dots,\Delta_{k}^{N}B), \quad k=1,\dots,N.
\end{equation}
Hence, as previously discussed, we just need to simulate the path of a Brownian motion for this simulation scheme. We summarize the methodology in the following algorithm.
\begin{algorithm}[H] 
\SetAlgoLined
    \textbf{Input:} $N\in\mathbb{N}$, $T>0$, $b:[0,T]\times \mathbb{R}^{d}\to\mathbb{R}^{d}$, $\sigma:[0,T]\times\mathbb{R}^{d}\to\mathbb{R}^{d\times m}$, $x_{0}\in\mathbb{R}^{d}$.\newline
    \textbf{Output:} Approximation $(X_{t_k}^{N})_{k=0,\dots,N}.$\newline
    $t_{k}\gets k\Delta_{N}$, $k=0,1,\dots,N$ with $\Delta_{N}=T/N$; \Comment{Define the grid.}\newline
    $B=(B_{t_k})_{k=0,\dots,N_{n}}$;\Comment{Simulate a Brownian Motion.}\newline
    $X_0^{N}\gets x_0$;\Comment{Initialization.}\newline
    \For{$k=1 \textbf{\textrm{ to }} N$}
        {$X_{t_k}^{N}\gets X_{t_{k-1}}^{N} + b(t_{k-1},X_{t_{k-1}}^{N})\Delta_{N} + \sigma(t_{k-1},X_{t_{k-1}}^{N})\Delta_{k}^{N}B$;}
\caption{Euler-Maruyama Scheme}
\label{alg:euler}
\end{algorithm}

\subsection{The Black \& Scholes Model}\label{sec:BSM}
Let $\mathfrak{M}=\{(\Omega,\mathcal{F},(\mathcal{F}_{t})_{0\leq t\leq T},\mathbb{P}), (A_{t},S_{t})_{0\leq t\leq T}\}$ with $T>0$ be our finite-horizon financial market consisting of a risky asset $S_{t}$ and a risk-free bond $A_{t}$ playing the role as a numéraire. The price dynamics specified by the Black \& Scholes Model are given by the pair of SDEs:
\begin{align}
    dS_{t}&=\mu S_{t}dt+\sigma S_{t}dB_{t},\\
    dA_{t}&=rA_{t}dt,
\end{align}
where $r>0$ is the risk-free rate, $(\mu,\sigma)\in\R\times (0,\infty)$ are constants, $A_{0}=1$, and $B_t$ is a Brownian motion. Obviously, the solution to (2.7) is given by $A_{t}=e^{rt}$ for $t\in [0,T]$.

Recall the (simple) Itô's lemma
\begin{equation}\label{itoslemma}
    f(t,B_{t})= f(0,B_{0})+\int_{0}^{t}\frac{\partial f}{\partial s}ds + \int_{0}^{t}\frac{\partial f}{\partial x}dB_{s}+\frac{1}{2}\int_{0}^{t}\frac{\partial^{2}f}{\partial x^{2}}d\langle B\rangle_{s},
\end{equation}
where $f\in C^{1,2}(\R^{2},\R)$ is once and twice continuously differentiable in the first and second variable, respectively. We now define the function
\begin{equation}
    f(t,B_{t})\coloneqq S_{0}\exp\left\{\left(\mu-\frac{\sigma^2}{2}\right)t+\sigma B_{t}\right\},\quad S_{0}>0.
\end{equation}
We can now easily calculate the partial derivatives as $\partial_{t}f=(\mu-\sigma^{2}/2)f$, $\partial_{x}f=\sigma f$, and $\partial^{2}_{x}f=\sigma^{2}f$. Furthermore, we have $f(0,B_{0})=S_{0}$. Inserting this into Itô's lemma, we obtain
\begin{align}
    f(t,B_{t})&=S_{0} + \left(\mu-\frac{\sigma^{2}}{2}\right)\int_{0}^{t}f(s,B_{s})ds+\sigma\int_{0}^{t}f(s,B_{s})dB_{s}+\frac{\sigma^{2}}{2}\int_{0}^{t}f(s,B_{s})ds\\
    &= S_{0}+\mu\int_{0}^{t}f(s,B_{s})ds+\sigma\int_{0}^{t}f(s,B_{s})dB_{s},
\end{align}
where we have also used that $\langle B\rangle_{t}=t$. The above can be written equivalently as
\begin{equation}
    df=\mu fdt+ \sigma fdB_{t}.
\end{equation}
Hence, the process $S_{t}\coloneqq f(t,B_{t})=S_{0}\exp\left\{(\mu-\sigma^{2}/2)t+\sigma B_{t}\right\}$ solves the SDE (2.6). Such a process is called a geometric Brownian motion.

Consider a trading strategy in this market with continuous rebalancing, and which is self-financing, i.e. there are no exogenous incoming or outgoing cash flows. If we hold $\delta_{t}$ units of the risky asset (stock) $S_t$ for each point in time $t$, then the portfolio value changes as
\begin{equation}
    dV_{t}=\delta_{t}dS_{t}+(V_{t}-\delta_{t}S_{t})rdt,\quad 0\leq t\leq T.
\end{equation}
Now, consider a European option with maturity time $T$ and pay-off $\Phi(S_{T})$. We want to create a self-financing portfolio, which replicates the pay-off of the option
\begin{equation}
    V_{t}=F(t,S_{t})
\end{equation}
for some function $F$. By the no-arbitrage principle, our function $F$, which equals the replicating portfolio $V_t$, must be the price function of our option. If we apply Itô's lemma \eqref{itoslemma} to $V_{t}=F(t,S_{t})$ with $B_{t}=S_{t}$ we obtain
\begin{equation}\label{hejhej}
    dV_{t}=\delta_{t}dS_{t}+(V_{t}-\delta_{t}S_{t})rdt=\frac{\partial F}{\partial t}dt+\frac{\partial F}{\partial S_{t}}dS_{t}+\frac{1}{2}\frac{\partial^{2}F}{\partial S_{t}^{2}}d\langle S\rangle_{t}.
\end{equation}
The quadratic variation of a geometric Brownian motion is $\langle S\rangle_{t}=\sigma^{2}S_{t}^{2}t$. If we equate terms in \eqref{hejhej}, we get that \eqref{hejhej} holds if
\begin{equation}
    \delta_{t}=\frac{\partial F}{\partial S_{t}},\hspace{10 pt} r\left(F(t,S_{t})-\frac{\partial F}{\partial S_{t}}S_{t}\right)=\frac{\partial F}{\partial t}+\frac{1}{2}\sigma^{2}S_{t}^{2}\frac{\partial^{2}F}{\partial S_{t}^{2}}.
\end{equation}
Whence, we have derived the Black \& Scholes partial differential equation with boundary condition
\begin{equation}\label{blackscholespde}
    -rF(t,S_{t})+\frac{\partial F}{\partial t}+rS_{t}\frac{\partial F}{\partial S_{t}}+\frac{1}{2}\sigma^{2}S_{t}^{2}\frac{\partial^{2}F}{\partial S_{t}^{2}},\quad F(T,S_{t})=\Phi(S_{T}).
\end{equation}
The portfolio $V_{t}=\delta_{t}S_{t}+(V_{t}-\delta_{t}S_{t})r$ with $\delta_{t}=\partial_{S_{t}}F(t,S_{t})$ thus replicates the pay-off of the option. The quantity $\delta_{t}$ is called the delta of the option and measures the sensitivity of the option price to changes in the underlying stock price. Hence, the price function of the option $F$ must satisfy \eqref{blackscholespde}. 

For a European call option with strike price $K>0$ and pay-off $\Phi(S_{T})=(S_{T}-K)^{+}$, we obtain
\begin{align}
    F(t,S_{t})&=S_{t}N(d_{1})-e^{-r(T-t)}KN(d_{2}),\quad 0\leq t\leq T,\label{BSFormula}\\
    d_{1}&\coloneqq \frac{\ln\left(\frac{S_{t}}{K}\right)+\left(r+\frac{\sigma^{2}}{2}\right)(T-t)}{\sigma\sqrt{T-t}},\\
    d_{2}&\coloneqq d_{1}-\sigma\sqrt{T-t},
\end{align}
and where $N(\cdot)$ is the standard normal cdf. For a general simple derivative $\xi=\Phi(S_{T})$, we have the following result.
\begin{thm}[\textit{Price Function}]
    Let $\xi$ be a simple derivative, i.e. $\xi=\Phi(S_{T})$ for some measurable function $\Phi$. Then the function
    \begin{equation}
        F(t,x)\coloneqq e^{-r(T-t)}\mathbb{E}_{\mathbb{Q}}\left[\Phi(X(x,t,T))\right],
    \end{equation}
    where 
    \begin{equation}
        X(x,t,T)\coloneqq xe^{(r-\sigma^{2}/2)(T-t)+\sigma\sqrt{T-t}Z},\quad Z\sim N(0,1),
    \end{equation}
satifies almost surely that
\begin{equation}
    F(t,S_{t})=\mathbb{E}_{\mathbb{Q}}\left(\frac{A_{t}}{A_{T}}\xi \mid \mathcal{F}_{t}\right).
\end{equation}
\end{thm}
Before we prove Theorem 2.2, we present a lemma on conditional expectations.
\begin{lem}\label{lem23}
    Let $X,Y$ be two $d$-dimensional random vectors and $\mathcal{G}$ a sub-$\sigma$-algebra of $\mathcal{F}$. Suppose that $Y$ is $\mathcal{G}$-measurable, and $X$ is independent of $Y$. Then for any measurable function $f:\R^{d}\times \R^{d}\to\R$ satisfying that $\mathbb{E}(|f(X,Y)|)<\infty$, we have that almost surely
    \begin{equation}
\mathbb{E}(f(X,Y)\mid \mathcal{G})=g(Y),
    \end{equation}
    where $g(x)\coloneqq \mathbb{E}(f(X,x))$.
\end{lem}
\begin{proof}
    Using Girsanov's Theorem, one can show that in the Black \& Scholes model
    \begin{equation}
        S_{T}=S_{t}e^{(r-\sigma^{2}/2)(T-t)+\sigma(B_{T}^{\mathbb{Q}}-B_{t}^{\mathbb{Q}})}.
    \end{equation}
    Thus, we have
    \begin{equation}
        \mathbb{E}_{\mathbb{Q}}\left(\frac{A_{t}}{A_{T}}\xi \mid \mathcal{F}_{t}\right)=e^{-r(T-t)}\mathbb{E}_{\mathbb{Q}}\left[\Phi\left(S_{t}e^{(r-\sigma^{2}/2)(T-t)+\sigma(B_{T}^{\mathbb{Q}}-B_{t}^{\mathbb{Q}})}\right)\mid \mathcal{F}_{t}\right].
    \end{equation}
    Since $B_{T}^{\mathbb{Q}}-B_{t}^{\mathbb{Q}}$ is independent of $\mathcal{F}_t$, we can use Lemma \ref{lem23} to the case when $X=e^{(r-\sigma^{2}/2)(T-t)+\sigma(B_{T}^{\mathbb{Q}}-B_{t}^{\mathbb{Q}})}$, $Y=S_{t}$, $\mathcal{G}=\mathcal{F}_{t}$, and $f(x,y)=\Phi(xy)$ to obtain that
    \begin{equation}
        \mathbb{E}_{\mathbb{Q}}\left(\frac{A_{t}}{A_{T}}\xi \mid \mathcal{F}_{t}\right)=e^{-r(T-t)}g(S_{t}),
    \end{equation}
    where
    \begin{equation}
        g(x)=\mathbb{E}_{\mathbb{Q}}\left[\Phi\left(xe^{(r-\sigma^{2}/2)(T-t)+\sigma(B_{T}^{\mathbb{Q}}-B_{t}^{\mathbb{Q}})}\right)\right]=\mathbb{E}_{\mathbb{Q}}[\Phi(X(x,t,T))].
    \end{equation}
\end{proof}
\subsection{Implied Volatility}
If we return to the price function (2.18). Then we could reparameterize $F$, and consider it as a function of $(\tau, y, \sigma)$, where $\tau=T-t$ is time to maturity, $y=S_{t}/K$ is the so-called moneyness, and $\sigma$ is the volatility. These quantities are all observable except for $\sigma$. However, the mapping $\sigma\mapsto F(t,x,\sigma)$ is strictly increasing on $(0,\infty)$, and consequently also invertible. Hence, there is a one-to-one correspondence between the price of an option and the volatility of the underlying stock. 
\begin{defn}[\textit{Implied Volatility}]
    Let $F$ be the Black \& Scholes price function of a European call option. If $\Pi_{t}$ denotes the market price of a European call option with time to maturity $\tau=T-t$ and moneyness $y=S_{t}/K$, then the implied volatility of such a contract is defined as the unique number $\sigma_{BS}(\tau, y)>0$, which satisfies
    \begin{equation}
        \Pi_{t}=F\left(T-\tau, \hspace{1 pt}yK,\hspace{1 pt}\sigma_{BS}(\tau,y)\right).
    \end{equation}
\end{defn}
