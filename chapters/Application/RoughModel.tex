\chapter{Volatility Models}
In this chapter, we will...
\section{Rough Stochastic Volatility Models}
The motivation for the development of stochastic volatility models was the constant volatility assumption of the Black \& Scholes model, which proved to be incompatible with observed market option prices. Hence, stochastic volatility models must also specify the stochastics of the volatility process. One of the most popular stochastic volatility models is the Heston model, which is specified by the pair of SDEs
\begin{align}
    dS_{t}&= \mu S_{t}dt + \sqrt{\nu_{t}}S_{t}dB_{t}^{(1)},\\
    d\nu_{t}&= \kappa(\theta - \nu_{t})dt + \xi\sqrt{\nu_{t}}dB_{t}^{(2)},
\end{align}
where $\langle B^{(1)},B^{(2)}\rangle_{t}=\rho t$ with $|\rho|\leq 1$. Additionally, it is assumed that $S_{0},\nu_{0}>0$, and that the so-called Feller condition $2\kappa\theta >\xi^{2}$ is satisfied ensuring the volatility process is strictly positive.